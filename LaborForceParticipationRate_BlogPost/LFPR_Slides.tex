\documentclass[12pt]{beamer}
\usepackage{graphicx}
\usepackage{float}
\usepackage[T1]{fontenc}

\begin{document}
\begin{frame}
    \frametitle{Who Wins and Who Loses in the  Labor Market During Economic Downturns?}
\begin{itemize}
    \item Labor force participation rate (LFPR) refers to the number of people in the labor force as a percentage of the working-age civilian noninstitutional population.
\end{itemize}

    $$\text{LFPR} = \frac{\text{Employed People + Unemployed People}}{\text{Working Age Non-institutionalized Population}} * 100 \%$$
\begin{itemize}
    \item Economists use LFPR as an initial economic indicator of current market trends and to understand the overall health of the economy, with some exceptions, particularly in the long run.
\end{itemize}
\end{frame}

\begin{frame}
    \frametitle{Labor Force Participation Rate by Gender}

    \begin{columns}
        \column{0.5\textwidth}
        \centering
        \includegraphics[width=\linewidth]{lfpr_gender_1948_2025.jpg}
        
        \small Labor Force Participation Rate by Gender (1948-2025)

        \column{0.5\textwidth}
        \centering
        \includegraphics[width=\linewidth]{lfpr_gender_2000_2025.jpg}
        
        \small Labor Force Participation Rate by Gender (2000-2025)
    \end{columns}

    \bigskip

    \begin{itemize}
        \item Why the LFPR of female increased while decreased for male workers?
        \item Why the overall LFPR has been decreasing since 2000?
    \end{itemize}

\end{frame}

\begin{frame}
    \frametitle{Labor Force Participation Rate by Race}

    \begin{columns}
        \column{0.5\textwidth}
        \centering
        \includegraphics[width=\linewidth]{lfpr_race_1948_2025.jpg}
        
        \small Labor Force Participation Rate by Race (1948-2025)

        \column{0.5\textwidth}
        \centering
        \includegraphics[width=\linewidth]{lfpr_race_2000_2025.jpg}
        
        \small Labor Force Participation Rate by Race (2000-2025)
    \end{columns}


 \bigskip

\begin{itemize}
    \item Historical trend of LFPR during Great Recession and COVID-19
\end{itemize}

\end{frame}

\begin{frame}
    \begin{figure}
        \centering
        \includegraphics[width=0.65\linewidth]{lfpr_race_2014_2025.jpg}
        \caption{Labor Force Participation Rate by Race during COVID-19 (2014--2025)}  
    \end{figure}

\begin{itemize}
    \item Why the LFPR of Black and Hispanic increased while decreased for White after COVID-19?
    \item Why Black and Hispanic workers are more vulnerable to economic downturns?
\end{itemize}

\end{frame}

\end{document}