\documentclass[letterpaper]{scrartcl}
\usepackage{setspace}
\usepackage{amsmath}



% Citation BibLatex-APA
\usepackage[american]{babel}
\usepackage[utf8]{inputenc}
\usepackage{csquotes}
\usepackage[backend=biber, style=apa]{biblatex}
\addbibresource{references.bib}
%%%

\usepackage{lipsum}
\usepackage{graphicx}
\usepackage{float}
\usepackage{subcaption}
\usepackage[T1]{fontenc}

\usepackage[margin=1in]{geometry}
\usepackage{hyperref}
\usepackage{xcolor}

\definecolor{triton_green}{RGB}{4,106,56}


\titlehead{
\noindent
E2S2
\begin{tabular}[t]{@{}l}
  	\hspace{4in}\large \textbf{Black Wealth Data Center} \\
\end{tabular}
\hfill
\hrule
}


\begin{document}
    \onehalfspacing
    \title{Who Wins and Who Loses in the  Labor Market During Economic Downturns?}
    \author{Subodh Pandey}
    \date{}

	\maketitle



Labor force participation rate (LFPR) refers to the number of people in the labor force as a percentage of the working-age civilian noninstitutional population. Broadly, LFPR  can be defined as the percentage of the working age population that is in the civilian labor force. Civilian labor force is a term used by the United States Bureau of Labor Statistics (BLS) to describe the subsets of the American population that are: 1) at least 16 years old 2) not serving in the military and 3) not institutionalized (e.g., not in penal and mental facilities, not in aged-homes), and who are not on active duty in the Armed Forces. The LFPR is estimated by the BLS from the responses to the Current Population Survey.

The LFPR is calculated using the following formula:

$$\text{LFPR} = \frac{\text{Employed People + Unemployed People}}{\text{Working Age Non-institutionalized Population}} * 100 \%$$

The study of the trend in LFPR is important due to several reasons.  It represents the relative size of the labor resources that is available in the production of the Nation's good and services \parencite{mosisa2006trends}. Also the civilian non institutional population of the United States in 2025 was approximately 273 million. The movement in 1 percentage points LFPR involves a change in the labor market status of 2.73 million people.

Macroeconomists use LFPR as an initial economic indicator of current market trends and to understand the overall health of the economy, with some exceptions, particularly in the long run. The increase in the labor force growth is associated with faster GDP growth, and vice versa. However, long-run changes may not always reflect the health of the economy. For example, demographic changes such as an aging population or the choice of individuals to attend college rather than work may lead to more exits from the labor force, causing a shrinking of the labor force and thus decreasing the LFPR.

While looking at the historical trend, the LFPR reached a peak of 67.3\% in early 2000. The rate followed an upward trend for several decades since the data collection started in 1948.  \textcite{perez2021decline} attributed the decline to technological innovations and a change in the social programs after the financial crisis of 2007. \textcite{krueger2017have} attributed the decline to the retirement of the baby boomers, increase in social security benefits, changes to employed provided private insurance \parencite{mosisa2006trends}, and increased school enrollment among the younger population.

\begin{figure}[H]
    \centering
    \begin{subfigure}[t]{0.49\textwidth}
        \centering
        \includegraphics[width=\linewidth]{lfpr_gender_1948_2025_rec.jpg}
        \caption{Labor Force Participation Rate by Gender (1948--2025)}
        \label{fig:lfpr_1984_2025}
    \end{subfigure}
    \hfill
    \begin{subfigure}[t]{0.49\textwidth}
        \centering
        \includegraphics[width=\linewidth]{lfpr_gender_2000_2025_rec.jpg}
        \caption{Labor Force Participation Rate by Gender (2000--2025)}
        \label{fig:lfpr_2000_2025}
    \end{subfigure}

    \caption{Trends in Labor Force Participation Rate by Gender. 
    Shaded areas indicate U.S. recession periods.}
    \label{fig:lfpr_combined}
\end{figure}

The LFPR of men has been persistently higher than that of women since the data collection started. From then the it fell by 7.4 points among men and grew by 2.8 points among women, comparing 2025 December with 2000 January. While comparing 2025 December with 1948 January, it fell by 19 points among men and grew by 25.3 points among females.

The LFPR has changed by gender centered around evolving societal norms around gender roles \parencite{Fu2025LFPR}. Historically, women experienced a significant increase in LFPR, particularly from the 1950s to the 1990s, while men experienced a decline in LFPR during the same period. However, since the early 2000s, the LFPR for both male and female has been declining parallelly. Though men have a higher LFPR than women, the gap has narrowed over time from 54.7 percentage points on January 1948 to 10.4 percentage points on December 2025.

The increase in the LFPR for female can be attributed to several factors, such as the feminist movement, increase in post marriage work culture , declined fertility, delayed marriage, high divorce rates, increased access to education and employment opportunities, changes in societal norms, acceptance of women in the labor market, increased demand of women in the service sector, development of labor saving household technology, changes in husbands income, and the implementation of policies that support work-life balance \parencite{killingsworth1986female,greenwood2005engines, juhn2006changes,salamaliki2014smooth}. 

On the other hand, the decline in the LFPR for men can be attributed to several factors, such as steep decline in the participation of older men near or above 65 due to introduction of social security and early retriment of older workers, decline in demand for less skilled workers, decline of manufacturing jobs, increased educational enrollment, and changes in societal norms around gender roles \parencite{murphy1997unemployment,juhn2006changes,salamaliki2014smooth}.

The efects of economic downturns on the LFPR are also different by gender. During the COVID-19 pandemic, the LFPR for women dropped more significantly than for men, as women were more likely to work in industries severely affected by the pandemic, such as hospitality and retail.

\begin{figure}[H]
    \centering
    \begin{subfigure}[t]{0.49\textwidth}
        \centering
        \includegraphics[width=\linewidth]{lfpr_race_2000_2025_rec.jpg}
        \caption{Labor Force Participation Rate by Race (2000--2025)}
        \label{fig:lfpr_race_2000_2025}
    \end{subfigure}
    \hfill
    \begin{subfigure}[t]{0.49\textwidth}
        \centering
        \includegraphics[width=\linewidth]{lfpr_race_2014_2025_rec.jpg}
        \caption{Labor Force Participation Rate by Race (2014--2025)}
        \label{fig:lfpr_race_2014_2025}
    \end{subfigure}

    \caption{Trends in Labor Force Participation Rate by Race. 
    Shaded areas indicate U.S. recession periods.}
    \label{fig:lfpr_combined}
\end{figure}

Historically, the LFPR for White workers has been higher than that of Black workers and LFPR for the Hispanic/Latino is higher than Whites and Blacks. The gap between these workers has remained relatively stable. However, the LFPR for Black workers has been more volatile than that of White and Hispanic workers, with larger fluctuations during economic downturns.

The LFPR is procyclical-it increases during economic expansions and decreases during recessions. More individuals join the labor force when the jobs are plentiful and leave in response to the fewer job opportunities \parencite{mosisa2006trends}. The U.S. labor market was severely affected by the coronavirus disease 2019 (COVID-19) pandemic as the unemployment rate increased to 14.7 percent in April 2020 from 3.6 percent in April 2019. As 2020 continued, the labor market began to improve, and the unemployment rate started to fall. The labor force participation rate, however, did not recover as fast. This low participation rate during COVID-19 has been attributed to several factors, such as dependent care demands, increased unemployment benefits, people afraid of getting sick from COVID-19, and work-life balance preferences. Economists have also pointed to retirements, generous severance after a layoff, younger workers' focus on higher education, and general disillusionment with the labor market as possible reasons for the declining share of workers in the labor force, particularly among the youngest and oldest groups.

While evaluating the effects during the Great Recession, there was a steady decrease in the LFPR, which was a steady decrease that continued since 2000, and was further decreased after the recession. Unlike COVID-19, where millions "dropped out" due to health fears or lockdowns, the Great Recession saw a more gradual exit from the labor force as discouraged workers stopped looking for jobs over several years.

There is a differentiated effects of market shocks (COVID-19 and the Great Recession) on the LFPR by race. The LFPR during COVID-19 significantly reduced, while the reduction effects are higher in black (approx. 5\%) and Hispanic (approx. 5\%) as compared to Whites (3\%). The possible reasons are being an essential worker, the type of work performed, workplace factors, and community and geographic factors. 

Thus, health effects were prominent during the COVID-19 pandemic, resulting in the rapid exit, causing a huge drop in the labor force participation rate. While the Great Recession caused a gradual exit from the labor force, they stopped looking for a job, making a rebound difficult.



%%Extra
Based on geographical location, the top five states that have the highest labor force participation rate are North Dakota, Nebraska, Utah, South Dakota, and Minnesota. The five states that have the lowest labor force participation rate are West Virginia, Mississippi, Alabama, New Mexico, and South Carolina.

The reasons attributed to the reasons presented by Jason et al. (2023), which include: (1) being an essential worker, (2) the type of work performed, (3) workplace factors, and (4) community and geographic factors. 

\newpage
\printbibliography

\newpage
Notes
\begin{figure}[H]
        \centering
		\includegraphics[width=0.8\linewidth]{LaborForceParticipationRate (1).png}
        \caption{Labor Force Participation Rate by Race and Ethnicity}
        \label{fig1_lfpr_race_ethnicity}
        \begin{minipage}{15cm}
        \end{minipage}
\end{figure}

\begin{figure}[H]
    \centering
    \includegraphics[width=0.8\linewidth]{Screenshot 2026-01-20 at 11.42.09 AM.png}
    \caption{Labor Force Participation Rate during COVID-19 and The Great Recession}
    \label{fig:placeholder}
    \begin{minipage}{15cm}
    \end{minipage}
\end{figure}



\end{document}
