\documentclass[letterpaper]{scrartcl}
\usepackage{setspace}
\usepackage{amsmath}



% Citation BibLatex-APA
\usepackage[american]{babel}
\usepackage[utf8]{inputenc}
\usepackage{csquotes}
\usepackage[backend=biber, style=apa]{biblatex}
\addbibresource{references.bib}
%%%

\usepackage{lipsum}
\usepackage{graphicx}
\usepackage{float}
\usepackage[T1]{fontenc}

\usepackage[tmargin=0.5in,bmargin=0.5in, lmargin=0.5in, rmargin=0.5in]{geometry}
\usepackage{hyperref}
\usepackage{xcolor}

\definecolor{triton_green}{RGB}{4,106,56}


\titlehead{
\noindent
E2S2
\begin{tabular}[t]{@{}l}
  	\hspace{5in}\large \textbf{Black Wealth Data Center} \\
\end{tabular}
\hfill
\hrule
}


\begin{document}
    \onehalfspacing
    \title{Who Wins and Who Loses in the  Labor Market During Economic Downturns?}
    \author{Subodh Pandey}
    \date{}

	\maketitle
	



Labor force participation rate (LFPR) refers to the number of people in the labor force as a percentage of the working-age civilian noninstitutional population. Broadly, LFPR  can be defined as the percentage of the population that is in the civilian labor force. Civilian labor force is a term used by the United States Bureau of Labor Statistics (BLS) to describe the subsets of the American population that are: 1) at least 16 years old 2) not serving in the military and 3) not institutionalized (e.g., not in penal and mental facilities, not in aged-homes), and who are not on active duty in the Armed Forces.

$$\text{LFPR} = \frac{\text{Employed People + Unemployed People}}{\text{Working Age Non-institutionalized Population}} * 100 \%$$

Macroeconomists use LFPR as an initial economic indicator of current market trends and to understand the overall health of the economy, with some exceptions, particularly in the long run. The increase in the labor force growth is associated with faster GDP growth, and vice versa. However, long-run changes may not always reflect the health of the economy. For example, demographic changes such as an aging population or the choice of individuals to attend college rather than work may lead to more exits from the labor force, causing a shrinking of the labor force and thus decreasing the LFPR.

While looking at the historical trend, the LFPR reached a peak of 67.3\% in early 2000. The rate followed an upward trend for several decades since the data collection started in 1948, which may be attributed to more women entering the workforce. The LFPR of men has been persistently higher than that of women since the data collection started. From then the it fell by 7.4 points among men and grew by 2.8 points among women, comparing 2025 December with 2000 January. While comparing 2025 December with 1948 January, it fell by 19 points among men and grew by 25.3 points among females. \textcite{perez2021decline} attributed the decline to technological innovations and a change in the social programs after the financial crisis of 2007. \textcite{krueger2017have} attributed the decline to the retirement of the baby boomers, increase in social security benefits including Social Security Disability Insurance (SSDI) and increased school enrollment among the younger population.

The U.S. labor market was severely affected by the coronavirus disease 2019 (COVID-19) pandemic as the unemployment rate increased to 14.7 percent in April 2020 from 3.6 percent in April 2019. As 2020 continued, the labor market began to improve, and the unemployment rate started to fall. The labor force participation rate, however, did not recover as fast. This low participation rate during COVID-19 has been attributed to several factors, such as dependent care demands, increased unemployment benefits, people afraid of getting sick from COVID-19, and work-life balance preferences. Economists have also pointed to retirements, generous severance after a layoff, younger workers' focus on higher education, and general disillusionment with the labor market as possible reasons for the declining share of workers in the labor force, particularly among the youngest and oldest groups.

There are also varying effects of market shocks (COVID-19 and the Great Recession) on the LFPR. The LFPR during COVID-19 significantly reduced, while the reduction effects are higher in black (approx. 5\%) and Hispanic (approx. 5\%) as compared to Whites (3\%). The reasons attributed to the reasons presented by Jason et al. (2023), which include: (1) being an essential worker, (2) the type of work performed, (3) workplace factors, and (4) community and geographic factors. 

While evaluating the effects during the Great Recession, there was a steady decrease in the LFPR, which was a steady decrease that continued since 2000, and was further decreased after the recession. Unlike COVID-19, where millions "dropped out" due to health fears or lockdowns, the Great Recession saw a more gradual exit from the labor force as discouraged workers stopped looking for jobs over several years.

Thus, health effects were prominent during the COVID-19 pandemic, resulting in the rapid exit, causing a huge drop in the labor force participation rate. While the Great Recession caused a gradual exit from the labor force, they stopped looking for a job, making a rebound difficult. 

%%Extra
Based on geographical location, the top five states that have the highest labor force participation rate are North Dakota, Nebraska, Utah, South Dakota, and Minnesota. The five states that have the lowest labor force participation rate are West Virginia, Mississippi, Alabama, New Mexico, and South Carolina.

\newpage
\printbibliography

\newpage
Notes
\begin{figure}[H]
        \centering
		\includegraphics[width=0.8\linewidth]{LaborForceParticipationRate (1).png}
        \caption{Labor Force Participation Rate by Race and Ethnicity}
        \label{fig1_lfpr_race_ethnicity}
        \begin{minipage}{15cm}
        \end{minipage}
\end{figure}

\begin{figure}[H]
    \centering
    \includegraphics[width=0.8\linewidth]{Screenshot 2026-01-20 at 11.42.09 AM.png}
    \caption{Labor Force Participation Rate during COVID-19 and The Great Recession}
    \label{fig:placeholder}
    \begin{minipage}{15cm}
    \end{minipage}
\end{figure}



\end{document}
